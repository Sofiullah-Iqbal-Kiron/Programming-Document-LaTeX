\documentclass[11 pt]{book}

\usepackage[utf8]{inputenc}
\usepackage
{
 listings, %Code listing.
 xcolor, %Design with color.
 hyperref, %Add hyperlink / clickable link.
 graphicx, %Picture add and design with graphics.
 authblk, %Author affiliation.
 tikz,
 pgfplots, %Graph plots.
 tabularx, %Formatting table.
 lipsum,
 mversion %Adding version later of date.
}
\usepackage[skins]{tcolorbox} %Colored/nice textbox.
\usepackage[document]{ragged2e}
\usepackage[short, nodayofweek, level, 12hr]{datetime}
\usepackage[usestackEOL]{stackengine}

\setVersion{0.0}
\increaseBuild %Will update version at each recompilation.

\title{Programming Document}
\author
{
Sofiullah Iqbal Kiron\\
\href{mailto:sofiul.k.1023@gmail.com}{sofiul.k.1023@gmail.com}
}
\date{11 April, 2020 \\ \currenttime \\ Version: \version}
\affil{BSMRSTU, Department of CSE \\ SHIICT}

\lstset
{
 language=C++,
 backgroundcolor=\color{black!5},
 basicstyle=\footnotesize\ttfamily,
 keywordstyle=\color{blue},
 commentstyle=\color{green!80},
 showstringspaces=false,
 stringstyle=\color{red},
 captionpos=b
}

\renewcommand\contentsname{A-Z Menu}

\begin{document}

\pagecolor{green!90}
\maketitle
\pagecolor{white}
\tableofcontents
\listoffigures
\listoftables

{
	\raggedleft\vfill\itshape\Longstack[l] %You can change this properties to add more formation. As, \Longstack[l], \hfill, \raggedright, not \itshape, etc.
		{
			Thank You,\\
			Sofiullaha Iqbal Kiron.
		}\par
}

\chapter{Quotes}
\begin{flushleft}
	\begin{tcolorbox}[arc=3mm, width=6.8 cm, colback=red!5!white, colframe=red!75!black]
		Examples are better than $1000$ words.
	\end{tcolorbox}
\end{flushleft}

\begin{flushright}
	\begin{tcolorbox}[enhanced, title=Steve Jobs, attach boxed title to bottom right, boxed title style={arc=3mm, colframe=green}, drop large lifted shadow=blue!70, sharp corners]
		Everybody should learn how to program a computer because it teaches how to \textcolor{blue}{THINK}.
	\end{tcolorbox}
\end{flushright}

\begin{tcolorbox}[title=Winner, after title={\hfill\colorbox{blue}{Muhammad Ali}},
colback=red!5!white,colframe=red!75!black,fonttitle=\bfseries, sharp corners]
	I hated every minute of training. But I said, "don't quit, suffer now and live the rest of your life as a champion."
\end{tcolorbox}

\tcbset{frogbox/.style={enhanced,colback=green!10,colframe=green!65!black,
enlarge top by=5.5mm,
overlay={\foreach \x in {2cm, 3.5cm} {
\begin{scope}[shift={([xshift=\x]frame.north west)}]
\path[draw=green!65!black,fill=green!10,line width=1mm] (0,0) arc (0:180:5mm);
\path[fill=black] (-0.2,0) arc (0:180:1mm);
\end{scope}}}}}
\begin{flushright}
\begin{tcolorbox}[frogbox, arc=3 mm, width=4 cm]
You lost?
\end{tcolorbox}
\end{flushright}

\begin{tcolorbox}[width=7cm]
	Just practice and keep learning...
\end{tcolorbox}

\begin{flushright}
	\begin{tcolorbox}[title={Juma Ikangaa, marathoner}, colback=red!5!white, colframe=red!75!black, fonttitle=\bfseries]
		The will to win means nothing without the will to prepare.
	\end{tcolorbox}
\end{flushright}

%Chapter 2.
\chapter{C++, Problems and Algorithm}
\section{Evolution of C}
\begin{figure}[hbtp]
\centering
\includegraphics[width=200 px]{Timeline of C language.png}
\caption{Evolution Graph of C}
\end{figure}
C supports variable sized arrays from C99 standard.

\section{Array}
Here we will discuss about array data structure.\\
Reference: \href{https://www.geeksforgeeks.org/array-data-structure/}{GfG}\\

\subsection{Array rotation}
\subsection{Making subarray}
A subarray is a contiguous part of an array. An generated array that are already part of an another array.\\
e.g. : A given array - {1, 2, 3, 4}\\
Subarray of this array:
{1}, {2}, {3}, {4}, {1, 2}, {2, 3}, {3, 4}, {1, 2, 3}, {2, 3, 4}, {1, 2, 3, 4}
Given arrays are subarray of given array. A array holds $\frac{n(n+1)}{2}$ subarrays where $n$ is the number of elements in main array.

\section{Memory Allocation}
\subsection{Memory Layout of C}
Typical memory layout of C consists with following segments:-
\begin{enumerate}
	\item Text segment
	\item Initialized data segment
	\item Uninitialized data segment
	\item Stack
	\item Heap
\end{enumerate}
For better understanding, look at the block diagram below:-\\
\begin{figure}
	\centering
	\includegraphics[width=200px, height=140px]{C memory layout, GfG.png}
	\caption{C memory layout, GfG}
\end{figure}
\pagebreak

\textbf{Short explanation:}
\begin{itemize}
	\item[$\rightarrow$] \textbf{A text segment}, known as code segment or simply text holder, is one of the section of a program that contains executable instructions as a text. As a memory region, a text segment may be placed below the heap or stack in order to prevent heaps and stack overflows from overwriting it.
	\item[$\rightarrow$] \textbf{Initialized data segment} usually called \textbf{Data Segment (\textcolor{red}{DS}}), is a portion of virtual address space of a program, which contains \textit{global variables} and \textit{static variables} that are initialized by the programmer.
	\item[$\rightarrow$] \textbf{Uninitialized data segment}, generally known as \textbf{\textcolor{red}{BSS}}, named after an ancient assembler operator that stood for "\textit{block started by symbol}". It contains all global variables and static variables that are initialized to zero or do not have explicit initialization in source code.
	\item[$\rightarrow$] \textbf{Stack} is the segment for storing non-static and local variables.
	\item[$\rightarrow$] \textbf{Heap} is the segment where dynamic memory allocation usually takes place. The heap area began at the end of the BSS segment and grows to larger addresses from there.
\end{itemize}
Wanna learn more?, follow the \href{https://www.geeksforgeeks.org/memory-layout-of-c-program/}{LINK}.

\subsection{Pointers}
\href{https://www.youtube.com/watch?v=f2i0CnUOniA&list=PLBlnK6fEyqRjoG6aJ4FvFU1tlXbjLBiOP&index=18&t=0s}{NESO Academy}\\
Pointers is a special type of variables which is capable to store initial address of a variable which is points to. That's mean, $x$ is a int variable and that takes byte in memory of the location at 1002 and 1003. Then a pointer that points $x$ will return 1002, the initial point.\\
General syntax for declaring pointer:-\\
\textcolor{red}{data\_type *pointer\_name;}\\
\begin{center}
	\includegraphics[width=180 pt]{Pointer which is points to.png}
\end{center}
We can assign values to the pointer by \textcolor{red}{address\_of} operator. Which is \textcolor{red}{Ampersand}.
\begin{center}
	\includegraphics[width=180 pt]{Assigning address to the pointer.png}
\end{center}
We know that, pointer is also a variable so that it also takes place in memory and then store address of another variable in it.

\subsection{Dynamic Memory Allocation}
Dynamic memory allocation in C/C++ refers to performing allocate memory manually by the programmer. Dynamically allocated memories allocated on \textbf{HEAP}.\\
\begin{tcolorbox}
	Dynamic memory allocation is the process of assigning memory cell during execution time of the program.
		\begin{tcolorbox}[width=6.5cm, colframe=red!90]
			Run-Time memory allocation.
		\end{tcolorbox}
\end{tcolorbox}

\subsubsection{New and Delete operations}
Where C uses \textcolor{red}{malloc(}) function for dynamically allocate memory and free() for remove this. \textcolor{red}{malloc()} is a system function which allocates memory on the heap and returns a pointer to the new block. C++ standard supports these functions but also have two extra operator namely new and delete for performing same task in a better and easier way.\\
\textbf{New} is an operator which denotes a request to memory for allocate space on Heap. \textbf{New} operators initializes the memory and returns address of the newly allocated and initialized memory to the pointer variable.

\section{Illustration of C++ code}
 If there some problem with the STL of C++, then made own build in functions.
 Text before \dots
 \LaTeX
 \begin{lstlisting}
 for(int i=0; i<iterations; i++)
 {
  do something;
 }
 \end{lstlisting}
 Text after it \dots
 \subsection{Finding maximum element from an array of size n}

\begin{lstlisting}[frame=shadowbox, rulesepcolor=\color{blue!80}]
int MaxEle(int arr[], int n)
{
    int i, a=0;
    for(i=0; i<n; i++)
    {
        a = max(a, arr[i]);
    }
    return a;
} 
\end{lstlisting}

\subsection{Counting selected element from an array}
 
\begin{lstlisting}[frame=shadowbox, rulesepcolor=\color{green!80}]
int Count(int arr[], int n, int value)
{
    int c=0, i;
    for(i=0; i<n; i++)
    {
        if(arr[i]==value)
        {
            c++;
        }
    }
    return c;
}
 \end{lstlisting}
 
\subsection{Divide function by string}

\begin{tcolorbox}
\begin{lstlisting}
#include <bits/stdc++.h>
using namespace std;

string divide(string &ns, int &dividor, int &rem)
{
    int i;
    string result;
    rem = ns[0] - '0';
    for (i = 1; i < ns.size(); i++)
    {
        if (rem < dividor)
        {
            rem = rem * 10 + (ns[i] - '0');
            i++;
        }
        result.push_back(rem / dividor + '0');
        rem %= dividor;
    }
    return result;
}

int main()
{
    string ns;
    int dividor, rem;
    rem = 0;
    cin >> ns >> dividor;
    cout << ns << "/" << dividor << " = "
         << divide(ns, dividor, rem) << endl;
    cout << "Reminder: " << rem << endl;
}
\end{lstlisting}
\end{tcolorbox}

\subsection{BitWise Operator}
Works at bit-level.
Do we know that? Every odd number in binary representation hold true bit in first and last bit.
\begin{table}
\caption{Odd numbers in binary}
 \begin{center}
  \begin{tabular}{c c}
  ODD Decimal & in Binary\\
  1 & 1\\
  3 & 11\\
  5 & 101\\
  7 & 111\\
  9 & 1001 
  \end{tabular}
 \end{center}
\end{table}
Is it clear? OK. As reverse way, all even number in binary representation hold first true bit and false in last. For the sake of illustration:-
\begin{table}
\caption{Even numbers in binary}
 \begin{center}
  \begin{tabular}{c c}
  EVEN Decimal & in Binary\\
  2 & 10\\
  4 & 100\\
  6 & 110\\
  8 & 1000\\
  10 & 1010 
  \end{tabular}
 \end{center}
\end{table}

Bitwise operators works on binary bits of given number.
\subsubsection{Some Interesting things about bitwise operator}
\begin{enumerate}
 \item 'N' is a given number, N will be odd if bitwise operation between N and 1 is 1. Means: (N\&1)=1, if N if odd. Else 0 if N is even.
 \item The left shift and right shift operators should not be used for negative numbers.
 \item We can find odd occurring number in an array by XOR.\\
 \begin{lstlisting}
 s;kldf
 \end{lstlisting}
\end{enumerate}

\subsection{Structure}
\begin{lstlisting}[basicstyle=\footnotesize, frame=single, frameround=tftf, numbers=left, stepnumber=2, numberfirstline=false, caption=Structure in C/C++]
#include<bits/stdc++.h>
using namespace std;

///Now we gonna learning structure in C++.
/*
    Sometimes we need a group of different types
    data in one specific class collection. For get
    released from this problem, C and C++ provides
    a new user-defined data-type.
    To define a structure, use the keyword "struct".
*/
struct student
{
    /*
        All members of struct is generally public.
    */
    int ID;
    string sex;
};

int main()
{
    /*
        Let's declare a student struct of class 6 that
        has 3 students.
        First student(class6[0]) has 33 as roll.
    */
    student class6[3];
    class6[0].ID=33;
    class6[1].ID=34;
    cout << class6[0].ID << endl;
    cout << class6[1].ID << endl;
    student kiron;
    cin >> kiron.ID;
    cout << "Kirons id " << kiron.ID << endl;
    cin >> kiron.sex;
    cout << "Kirons sex " << kiron.sex << endl;

    /*
        Let's declare a pointer of type student.
        That can store address of student type variables.
    */
    student *ptr;
    /*
        Now This pointer will store address of kiron's all
        member function.
    */
    ptr = &kiron;
    /*
        Now ptr points to all member variable of
        struct variable kiron.
    */

    kiron.ID=65;
    ptr->sex="Male";
    cout << ptr->ID << endl;
    cout << kiron.sex << endl;
    /*
    	Arrow operator is used for access by pointer.
    */
}
\end{lstlisting}

\subsection{How to check efficiently that a number is palindrome or not}
\href{https://leetcode.com/articles/palindrome-number/}{Link}\\
\begin{flushleft}
\textbf{Intuition}
The first idea that comes to mind is to convert the number into string, and check if the string is a palindrome, but this would require extra non-constant space for creating the string which is not allowed by the problem description.\linebreak
\linebreak
Second idea would be reverting the number itself, and then compare the number with original number, if they are the same, then the number is a palindrome. However, if the reversed number is larger than \textbf{int.MAX}, we will hit integer overflow problem.
\linebreak
\linebreak
Following the thoughts based on the second idea, to avoid the overflow issue of the reverted number, what if we only revert half of the \textbf{int} number? After all, the reverse of the last half of the palindrome should be the same as the first half of the number, if the number is a palindrome.
\linebreak
\linebreak
For example, if the input is 1221, if we can revert the last part of the number "1221" from "21" to "12", and compare it with the first half of the number "12", since 12 is the same as 12, we know that the number is a palindrome.
\end{flushleft}
\textbf{Algorithm}\\
First of all we should take care of some edge cases. All negative numbers are not palindrome, for example: -123 is not a palindrome since the '-' does not equal to '3'. So we can return false for all negative numbers.
\linebreak
\linebreak
Now let's think about how to revert the last half of the number. For number 1221, if we do 1221 \% 10, we get the last digit 1, to get the second to the last digit, we need to remove the last digit from 1221, we could do so by dividing it by 10, 1221 / 10 = 122. Then we can get the last digit again by doing a modulus by 10, 122 \% 10 = 2, and if we multiply the last digit by 10 and add the second last digit, 1 * 10 + 2 = 12, it gives us the reverted number we want. Continuing this process would give us the reverted number with more digits.
\linebreak
\linebreak
Now the question is, \textit{how do we know that we've reached the half of the number?}
\linebreak
\linebreak
Since we divided the number by 10, and multiplied the reversed number by 10, when the original number is less than the reversed number, it means we've processed half of the number digits.

 \section{STL Map}
 Map is a associated container of C++ STL(Standard Template Library). A map variable has two part,
 
 \begin{enumerate}
  \item Key
  \item Value
 \end{enumerate}
 
 \href{https://www.geeksforgeeks.org/map-associative-containers-the-c-standard-template-library-stl/}{\textcolor{blue}{Geeks for Geeks}}: Maps are associative container that stores data in a mapped fashion. No two map values can have same key values.\\
 
 \begin{lstlisting}
  #include<map>
  // Declaring Map
  map<int, int> mp; // mp: map name.
  // First one is keyvalue, second one is mapvalue.
  
  // Insert elements in random order.
  mp.insert(pair<int, int>(1, 40));
  
  //Creating map iterator:
  map<int, int> :: iterator it;
  
  //Accessing the map data:
  it -> first; // Refer keyvalue.
  it -> second; // Refer mapvalue.
 \end{lstlisting}
 
  \subsection{Map member function}
   1. insert() :- The key must be unique.
   
   \begin{lstlisting}
   //Syntax:
    map_name.insert({key, value});
    
    //e.g.
    map<int, int> mp;
    mp.insert({1, 40});
   \end{lstlisting}
   
   2. count() :-
      
   \section{STL pair}
   
   \begin{lstlisting}
#include<iostream>

/*For pair access, include this header file.
utility is a STL.*/
#include<utility>
using namespace std;

int main()
{
    pair<int, int>p1;
    p1.first = 1;
    p1.second = 2;
    cout << p1.first << " " << p1.second << endl;

    ///Declaring pair array with the size 4.
    pair<int, int>p[4];
    //All pair value assigned to 0 automatically.

    p[0].first=1;
    p[0].second=2;
    cout << p[0].first << " " << p[0].second << endl;

    /*Here we not assigned any value to the pair p[1] and p[2].
    So there will print 0*/
    cout << p[1].first << " " << p[1].second << endl;
    cout << p[2].first << " " << p[2].second << endl;

    /*
     Member function: make_pair().
     This member function assign values to pair directly.
     Thats mean, this is more sexy.
     Syntax: pair_name = make_pair(value1, value2);
    */
    pair<string, double> ps;
    ps = make_pair("My S.S.C result: ", 4.56);
    cout << ps.first << ps.second << endl;

    /*
     Operating pairs.
     1. (==) comparison.
        It will return 1 is those pairs both values are equal otherwise 0.
        As same for all other comparisons like >=, <=, != etc.
    */
    pair<int, int> pair1, pair2, pair3;
    pair1 = make_pair(1, 4);
    pair2 = make_pair(1, 4);
    pair3 = make_pair(2, 4.1);
    /*No matter what type value we assigned,
    It will type casted automatically by the data type as we declared.
    So it will be same as (2, 4).
    */

    if(pair1==pair2)
    {
        cout << "Pairs are same" << endl;
    }
    cout << (pair1==pair2) << endl;
    if(pair1!=pair3)
    {
        cout << "pair1 and pair3 are not same" << endl;
    }
    else
    {
        cout << "pair1 and pair3 are same" << endl;
    }

    /*
     swap()
     Syntax: pair1.swap(pair2);
     It will swap values of both pair.
     swap pair1.first with pair2.first then
     swap pair1.second with pair2.second
    */
    pair1.swap(pair3);
    cout << pair1.first << " " << pair1.second << endl;
    
    /*
     swap() isn't working in code::blocks but in other
     online compiler, it works perfectly.
     */
}


   \end{lstlisting}
   
\section{STL List}
Lists are sequence of elements stored in a linked list. Compared to vectors, they allow fast insertions and deletions, but slower random access.\\
\subsection{Member Functions of List}
\begin{tabular}{|c c|}
\hline
Function & Task\\
\hline
assign & assign elements to list\\
\hline
begin & returns an iterator to the beginning of the list\\
\hline
back & returns a reference to last element of list\\
\hline
clear & removes all elements from the list(Clears the list)\\
\hline
empty & returns true if list is empty\\
\hline
end & returns an iterator just past the last element of a list\\
\hline
erase & remove specific element from list\\
\hline
front & returns a reference to the first element of a list\\
\hline
insert & insert elements into the list\\
\hline
max\_size & returns the maximum number of elements that the list can hold\\
\hline
merge & merge two lists\\
\hline
push\_back & add an element at end of the list\\
\hline
push\_front & add an element at beginning of the list\\
\hline
pop\_back & removes the last element of the list\\
\hline
pop\_front & remove first element of the list\\
\hline
rbegin & return reverse begin iterator\\
\hline
rend & return reverse end iterator\\
\hline
remove & remove elements from list\\
\hline
remove\_if & remove element conditionally\\
\hline
resize & change size of the list\\
\hline
reverse & reverse the list\\
\hline
size & returns the numbers of elements present in the list\\
\hline
sort & sorts the list in ascending order\\
\hline
splice & merge the lists in constant time\\
\hline
swap & swap two list with each other with all elements\\
\hline
\textcolor{red}{unique} & removes duplicate element\\
\hline
\end{tabular}
\subsection{Discuss}
\begin{lstlisting}
vector<int> v(5, 42);
\end{lstlisting}
It will produce a vector named v with five times 42.\\
\begin{lstlisting}[frame=shadowbox, rulesepcolor=\color{green!60}]
#include<bits/stdc++.h>
using namespace std;

int main()
{
    vector<int> v(5, 42);
    for(int i=0; i<v.size(); i++)
    {
        cout << v[i] << " ";
    }
    cout << endl;
}
\end{lstlisting}
\begin{tcolorbox}[colback=black!3]{Output:}
42 42 42 42 42
\end{tcolorbox}
\subsection{assign}
assign member function is used assign values from one to another.\\
\begin{lstlisting}[frame=shadowbox, rulesepcolor=\color{green!60}]
#include<bits/stdc++.h>
using namespace std;

int main()
{
    vector<int> v1;
    for(int i=0; i<10; i++)
    {
        v1.push_back(i);
    }
    vector<int> v2;
    v2.assign(v1.begin(), v1.end());
    for(int i=0; i<v2.size(); i++)
    {
        cout << v2[i] << " ";
    }
    cout << endl;
}
\end{lstlisting}

\section{STL Stack}
\begin{lstlisting}[caption=Stack, frame=shadowbox, rulesepcolor=\color{green!70}]
#include<iostream>
#include<stack>
using namespace std;

/*
 Stacks are a type of container adaptors with LIFO(Last IN First Out).
 So we can say, stack is Stup.
*/

int main()
{
    /*Let's create a stack of int type.*/
    stack<int> s;

    /*Member function push() will add the new element at the top
    of the stack.
    Time complexity O(1)*/
    s.push(10);
    s.push(9); /*9 stored above of 10*/
    s.push(8);
    s.push(1); /*Now 1 is the top most element.*/

    /*Algorithmic function size() returns current size of
    this stack.*/
    cout << s.size() << endl;

    /*Member function top() will return topper element or that element
    we entered last.
    For this example, it is 1*/
    cout << s.top() << endl;

    /*Member function pop() will delete the topmost element.*/
    s.pop();
    
    /*After removing topmost element,
    present topmost element is 8*/
    cout << s.top() << endl;
    
    /*After removing one element,
    current size of stack is 3*/
    cout << s.size() << endl;

    /*Algorithmic function empty() will return true if there is
    no element at the stack.*/
    if(!s.empty())
    {
        cout << "Stack is not empty" << endl;
    }

    return 0;
}
\end{lstlisting}
\begin{center}
\begin{tcolorbox}[enhanced, title=Output,
attach boxed title to bottom center, width=3.9 cm]
4\\1\\8\\3\\Stack is not empty
\end{tcolorbox}
\end{center}
   
\section{Power of 2}
\subsection{Binary Transform of \textbf{$2^n$}}
Look at the table:-\\(\LaTeX  provides a large set of tool for formatting tables)\\

\begin{center}
\begin{tabular}{||c | c | c | c||}
\hline
$2^n$ & Decimal & Binary & Seems like\\
\hline\hline
$2^0$ & 1 & 1 & $10^0$\\
\hline
$2^1$ & 2 & 10 & $10^1$\\
\hline
$2^2$ & 4 & 100 & $10^2$\\
\hline
$2^3$ & 8 & 1000 & $10^3$\\
\hline
$2^4$ & 16 & 10000 & $10^4$\\
\hline
$2^5$ & 32 & 100000 & $10^5$\\
\hline
\end{tabular}
\end{center}

It's clear that power of 2 in binary holds just one and only one true bit. So, we can simply say that, a number will be power of 2 if its binary holds just one true bit at first.

\subsection{Algorithm}
\begin{tcolorbox}
\begin{enumerate}
\item Take a decimal number string "ds".
\item Convert the number into binary string by a function.
 \begin{enumerate}
 \item Return binary string name such as "bs".
 \end{enumerate}
\item Start checking from the index 1.
\item If find out a true bit
 \begin{tcolorbox}
 return false;
 \end{tcolorbox}
 That mean, this is not power of 2.
\item Else: Till to end, there is no true bit.
 \begin{tcolorbox}
 return true;
 \end{tcolorbox}
 That mean, this is power of 2.
\end{enumerate}
\end{tcolorbox}

\section{Linked Lists}
Reference: \href{https://www.youtube.com/watch?v=njTh_OwMljA}{\textcolor{red}{Hacker Rank}}\\
Today we are going talk about linked lists. It's essentially just a sequence of the element, where each element links to the next elements which next element links to the next elements. A linked list can contain pretty much any kind of data - string, char, int. The elements can be sorted or unsorted, duplicates or unique. If we want to access element from linked list, we need linear time where array give us constant time, because for access array element, just boom, instantly do that. There are two type of linked list. Singly linked list and doubly linked list. Doubly linked list is a alternate version of singly linked list and access of elements from doubly link list is pretty much easy. One doubly linked list element has two part. One is previous node a second is next element. That mean, each elements having a link to the next element, each elements also linked to the previous element. So, for certain operations, this can be quite handy.\\
\href{https://www.geeksforgeeks.org/linked-list-set-1-introduction/}{\textcolor{red}{Geeks for Geeks}}\\
Advantages over arrays:-
\begin{itemize}
	\item Dynamic size
	\item Ease of insertion and deletion
\end{itemize}
Drawbacks:-
\begin{enumerate}
	\item Random access not allowed. We have to access elements sequentially from the first node called HEAD. So we can't do binary search with linked lists efficiently with its default implementation.
	\item Extra memory allocation required for a pointer with each elements of the list.
	\item As array, elements are not stored in contiguous location.
\end{enumerate}
Representation:-\\
A linked list is represented by a pointer to the first node of the linked list. If the linked list is empty, then the value of the HEAD is NULL. A node consist with two parts:-
\begin{itemize}
	\item Data
	\item Reference to the next node, i.e. pointer*.\footnote{\texttt{\hspace{0.4 cm}*A pointer is a special type of variable that stores memory address of another variable. Pointer is declared by asterisk sign before which variable we want to declare that it is a pointer. We can access address of a variable by using ampersand sign before it.}}
\end{itemize}

Let's create a user-defined data-type for build a node. A node has two part, one data and another pointer that points to the next node(That also has two part). So, the pointer must be such type that holds the next node(with two part). Means the pointer must be a node object.
IN C, node represented by struct. IN object-oriented language, it does with Class. Here the implementation in C++.
\begin{lstlisting}[caption=Node Class]
	class Node
	{
	 public:
	 	int data;
	 	Node *next;
	};
\end{lstlisting}
Block illustration of this code:-\\
\tikz
{
	\draw (0, 0)--(1, 0)--(1, -0.5)--(0, -0.5)--(0, 0);
	\draw (0.5, 0)--(0.5, -0.5);
	\draw[->] (1, -0.25)--(1.5, -0.25);
	\draw (1.5, 0)--(2.5, 0)--(2.5, -0.5)--(1.5, -0.5)--(1.5, 0);
	\draw (2, 0)--(2, -0.5);
	\draw[->] (2.5, -0.25)--(3, -0.25);
	\draw (3, 0)--(4, 0)--(4, -0.5)--(3, -0.5)--(3, 0);
	\draw (3.5, 0)--(3.5, -0.5);
}

 \section{Fortran}
 \textbf{Fortran} means Formula Translation. This is a general-purpose, compiled imperative programming language that's are specially suited to numeric computation and scientific computing, originally developed by IBM.\\
 (\textbf{IBM} - \textit{International Business Machines Corporation.})

\section{Python}
 In Linux and Mac operating systems, python are already installed. For install in windows, first we need to go \href{https://www.python.org/}{\textcolor{blue}{Pythons Official Website}}. Then download python and install.\\
\subsection{Why we use Python}
 Easy to learn, easy to use
 Popular in web development
 Can create a full web application
 
\chapter{Java}
Java is an open source high-level, general-purpose(Easy \& used to develop any kind of programs), object oriented language. If you wanna be a software developer than you must learn that 4 languages:-
\begin{enumerate}
	\item Python
	\item Java
	\item JavaScript
	\item C / C++
\end{enumerate}
In those, Java is second dominant (and popular also) language that you must learn if you wanna be a platform-independent developer. Also:-\hfill
\begin{itemize}
	\item[$\rightarrow$] Java has huge online community for getting help.
	\item[$\rightarrow$] Can be used in android development that is most preferable thing today.
\end{itemize}

Java has two types of error. Syntax and Semantic error. Syntax error is grammar error and semantic error is that the line has no meaning.
\begin{itemize}
	\item[$\rightarrow$] "I are playing" - this is syntax error.
	\item[$\rightarrow$] "He is hello" - this has no meaning, semantic error.
\end{itemize}
\begin{center}
	\includegraphics[width=280pt]{Editions of Java.png}
\end{center}

\section{IntelliJ IDEA}
IntelliJ IDEA is a greatest IDE for writing and compiling Java source codes. Here is some command and shortcuts of IntelliJ IDEA for windows operating system.

%\setlength{\arrayrulewidth}{1mm}
%\setlength{\tabcolsep}{18pt}
%\renewcommand{\arraystretch}{1.5}
%\begin{center}
	\begin{tabular}{| c | c |}
		\hline
		\multicolumn{2}{| c |}{IntelliJ IDEA}\\
		\hline
		Debug Program & shift + F9\\
		\hline
		Run Program & shift + F10\\
		\hline
		Format Codes & ctrl + alt + L\\
		\hline
		public static void main() & main / psvm\\
		\hline
		System.out.println() & sout\\
		\hline
		To warp a code block in a construct & ctrl + alt + T\\
		\hline
		Search anything in project & double click shift\\
		\hline
		Invoke commit changes dialog & ctrl + k\\
		\hline
		Select several words and edit together & press and hold shift + alt and double click on the word.\\
		\hline
		Fill the code construct & start typing method declaration and press ctrl + shift + enter\\
		\hline
		Join two statement in one line and remove unnecessary spaces & ctrl + shift + j\\
		\hline
		Jump highlighted syntax error & F2 / shift + f2\\
		\hline
		ctrl + shift + E & recently viewed or changed code fragment\\
		\hline
		Change name of the function from main method & highlight method name then press shift + f6 and re-write\\
		\hline
		Quick scheme & view | Quick Switch Scheme or ctrl + ˋ\\
		\hline
		To evaluate any expression while debugging program & select the expression in the editor and press Alt + F8\\
		\hline
		Create a new class & alt + ins\\
		\hline
		To see all the live template that are valid for this current method or context & ctrl + j (Close this by pressing "esc")\\
		\hline
		Complete syntax with dot from code completion helper(highlighted) & ctrl + dot\\
		\hline
		
	\end{tabular}
%\end{center}

\section{Syntax Difference between C++ and Java}
\setlength{\arrayrulewidth}{0.3 mm}
\setlength{\tabcolsep}{18pt}
\renewcommand{\arraystretch}{1.5}
\begin{center}
	\begin{table}
		\centering
		\begin{tabular}{| c | c |}
		\hline
		C++ & Java\\
		\hline
		string & String\\
		\hline
		int main() & public static void main()\\
		\hline
		empty() & isEmpty()\\
		\hline
		\end{tabular}
		\caption{Syntax Difference between C++ and Java}
	\end{table}
\end{center}

\chapter{JavaScript}
JavaScript is the programming language of HTML and web :\href{https://www.w3schools.com/js/}{w3schools}.\\
User rating of JS is $67.8\%$. See? how many people use JS. For become a great software developer or web developer, you must learn \textbf{JavaScript}. It seems impossible to be a developer without using JS. JavaScript provides an easy way to create interactive web pages smoothly.\\
All the functionality of JS in html written at $<$script$>$\dots$</$script$>$ tag.
\section{Why We Should Learn JS}
JavaScript is one of the 3 languages all web developer must learn.\\
\begin{enumerate}
 \item HTML
 \item CSS
 \item JavaScript(JS)
\end{enumerate}
Web pages are not the only place where \textbf{JS} is used. Many desktop and server programs use \textbf{JS}.
\section{Variable}
For declaring variable, use \textbf{\textcolor{red}{var}} keyword before variable name.\\
Syntax: \textcolor{red}{var variable\_name = value;}\\
If value is a string, write string value in single or double quotes. It's variable naming is same as C or C++. JavaScript is case sensitive. That's mean, var lastname and Lastname is not same.
\section{String}
JavaScript accepts both double and single quotes for assigning string values to variable.\\
var string;\\
string = "Kiron";\\
String operation as:\\
"Iqbal" + " " + "Kiron" is same as "Iqbal Kiron".
\section{Comments}
JS comment is fully same as C and C++ comments. JS supports both Oneline and Multiline comment.\\
Oneline comment: // or ///This is a comment.\\
Multiline comment: /*\dots*/

\chapter{Coursera Algorithm}
\begin{itemize}
	\item[Algorithm: ] Method for solving a problem.
	\item[Data Structure: ] Method to store information.
\end{itemize}
Algorithms of all of this course are implemented in Java.

\chapter{Windows Command Prompt}
Widows command prompt or windows terminal is a console for creating command line. If you wanna be a great developer then you must get used to uses of this.\\
\begin{center}
	\begin{tabularx}{0.9\textwidth}
	{|>{\raggedright\arraybackslash}X
	 |>{\raggedleft\arraybackslash}X	
	|}
		\hline
		{\color{red}Task}& {\color{red}Command Code}\\
		\hline
		\hline
		For clear total screen & cls\\
		\hline
		Will show all color code & color/?\\
		\hline
		For start a soft program & start soft\_name\\
		\hline
		Wanna exit from current folder? & exit\\
		\hline
		Open a file from current folder & cd folder\_name\\
		\hline
		View name of all files in current folder & dir\\
		\hline
		View name of all files in current folder with hidden files & dir/a\\
		\hline
		Make a new folder & mkdir folder\_name\\
		\hline
		Back from current folder & cd ..\\
		\hline
		Back more than one & cd ../.. as this\\
		\hline
		Remove directory or folder(If the folder is currently empty) & rmdir folder\_name\\
		\hline
		Remove directory or folder(If not empty) & rmdir /s folder\_name and then y or YES confirmation.\\
		\hline
		Compile java source code & javac file\_name.java\\
		\hline
		Run java class & java file\_name\\
		\hline
		Type few first words of file and then automatically filled full name & Press tab key :-not command\\
		\hline
		Delete all file as same type & del *.extension\\
		\hline
		
	\end{tabularx}
\end{center}

\chapter{Type Setting}
\section{Formatting}
Sometimes we need as special symbols that are not available in keyboard. Then go \href{https://www.tutorialspoint.com/online_latex_editor.php}{{\color{blue}here}}, click what symbol you need and copy the command from text editor of {\color{red}Tutorials Point}.

\section{Font}
Here the picture guide of Font-Size That we can use in "basicstyle" parameter at "lstset".
\includegraphics[scale=1]{../Font size.png} 

\section{Code Listing parameters}
Code listing parameters:\\
\includegraphics[scale=1]{../Code listing parameters.png} 

\section{Shadow frame in code listing}
Picture guide of shadow fraem in listing:\\
\includegraphics[scale=1]{../Shadow frame in code listing.png}

\section{Themes we can use in BEAMER class}
\subsection{Presentation themes without navigation bar}
\begin{enumerate}
 \item default
 \item boxes
 \item Bergen
 \item Boadilla
 \item Madrid
 \item AnnArbor
 \item CambridgeUS
 \item EastLansing
 \item Pittsburgh
 \item height=1 cm, Rochester
\end{enumerate}

\subsection{Presentation theme with a tree-like navigation bar}
\begin{enumerate}
 \item Antibes
 \item JuanLesPins
 \item Montpellier
 \item Berkeley
 \item Goettingen
 \item Marburg
 \item Hannover
 \item Berlin
 \item Ilmenau
 \item Dresden
 \item Darmstadt
 \item Frankfurt
 \item Singapore
 \item Szeged
\end{enumerate}

\section{\href{https://www.overleaf.com/learn/latex/List_of_Greek_letters_and_math_symbols}{List of Greek letters and math symbols}}
\begin{center}
	\includegraphics[width=200pt]{Greek Letters.png} 
	\includegraphics[width=200pt]{Arrows.png}
	\includegraphics[width=200pt]{Miscellaneous symbols.png}
	\includegraphics[width=200pt]{Binary Operation-Relation Symbols.png}
\end{center}

\begin{center}
	\begin{tcolorbox}[enhanced,
		size=minimal, auto outer arc,
		width=2.1cm, octogon arc,
		colback=red, colframe=white, colupper=white,
		fontupper=\fontsize{7mm}{7mm}\selectfont\bfseries\sffamily,
		halign=center, valign=center,
		square,arc is angular,
		borderline={0.2mm}{-1mm}{blue!70!green}]
		OVER
	\end{tcolorbox}
\end{center}

\end{document}